Para encontrar una solución al problema de esparcimiento y absorción de la luz por una partícula esférica, se considerará una región del espacio libre de fuentes y que los campos EM son armónicos en el tiempo, como se presentan en las Ecs. \eqref{} {\color{red} Isabel, pondrás las ecuaciones de Maxwell con su transformada de Fourier?}. Asimismo, se plantea un base de funciones vectoriales que permitan escribir a los campos EM como una combinación lineal de ellos.

Se propone un campo vectorial $\vb{M}$ tal que \cite{bohren1998absorption}
	\begin{align}
	\vb{M} &= \nabla \times \left(\vb{r} \psi\right),
	\label{eq:MrotCPsi}
	\end{align}
donde $\psi$ es una función escalar y $\vb{r}$ el vector de posición; dado que $\vb{M}$ es el rotacional de  $\vb{r}\psi$, se cumple que $\nabla\cdot \vb{M} = \vb{0}$. Asimismo, al calcular el rotacional de $\vb{r}\psi$, empleando la convención de la suma de Einstein y con $\epsilon_{ijk}$ el símbolo de  Levi-Civita, se obtiene que 
	\begin{align}
	M_i = [\nabla\times(\vb{r}\psi)]_i =  \epsilon_{ijk}\partial_j(r_k\psi) =\psi\epsilon_{ijk}\partial_j(r_k) -\epsilon_{ikj}r_k\partial_j\psi  =\psi[\nabla\times\vb{r}]_i - [\vb{r}\times\nabla\psi]_i = - [\vb{r}\times\nabla\psi]_i,
	\end{align}
es decir, que $\vb{M}$ y $\vb{r}$ son vectores perpendiculares.
 
  La  ecuación de Helmholtz para $\vb{M}$, dado que el operador laplaciano y el rotacional conmutan\footnote{ Para un campo vectorial arbitrario $\vb{A}$ se cumple que $\nabla^2\vb{A} = \nabla(\nabla\cdot\vb{A}) - \nabla\times(\nabla\times\vb{A})$, por lo que el rotacional del laplaciano de $\vb{A}$ es $ \nabla\times( \nabla^2\vb{A})=\nabla\times[\nabla(\nabla\cdot\vb{A})  ]-  \nabla\times[\nabla\times(\nabla\times \vb{A})] = -  \nabla\times[\nabla\times(\nabla\times \vb{A})] $ pues el rotacional del gradiente de cualquier función es nulo. Además, al sustituir $\vb{A}\to \nabla\times\vb{A}$ en la expresión del laplaciano de $\vb{A}$ y  considerando que la divergencia del rotacional de cualquier función es nulo, se obtiene que$ \nabla^2(\nabla\times\vb{A})=\nabla[\nabla\cdot(\nabla\times\vb{A})  ]-  \nabla\times[\nabla\times(\nabla\times \vb{A})] = -  \nabla\times[\nabla\times(\nabla\times \vb{A})] $. Por tanto, $\nabla^2$ y $\nabla\times$ con operadores que conmutan.}, es
	\begin{align}
	\nabla^2 \vb{M} + k^2 \nabla\vb{M} = \nabla\times \left[ \nabla^2\left(\vb{r} \psi\right)  
											+ k^2  \left(\vb{r} \psi\right) \right],
	\end{align}
y como   $\nabla^2 (\vb{r}\psi)=2\nabla\psi+\vb{r}\nabla^2\psi$, ya que
	\begin{align}
[\nabla^2 (\vb{r}\psi)]_i = \partial^2_{jj}(r_i\psi)= \partial_j [\partial_j(r_i)\psi+r_i\partial_j\psi] =\partial_{jj}{r_i} + 2 \partial_jr_i\partial_j\psi+r_i\partial^2_{jj}\psi,
	\end{align}
 donde $\partial_j r_i = \delta_{ij}$, con $\delta_{ij}$ la delta de Kronecker,  se cumple que $[\nabla^2 (\vb{r}\psi)]_i = 2\partial_i\psi+r_i\partial_{jj}\psi = 2[\nabla\psi]_i + [\vb{r}\nabla^2\psi]_i$, y por lo tanto $\nabla\times(\nabla \psi)=0$, la ecuación de Helmholtz para $\vb{M}$ puede reescribirse como
	\begin{align}
	\nabla^2 \vb{M} + k^2 \nabla\vb{M}  = \nabla\times\left[\vb{r}\left( \nabla^2\psi+k^2\psi \right) \right].
	\end{align}
Adicional a $\vb{M}$, se define el vector $\vb{N}$ como \cite{bohren1998absorption} 
	\begin{align}
	\vb{N} = \frac{\nabla\times \vb{M}}{k}, \label{eq:NrotM/k}
	\end{align}
cuyo laplaciano es $\nabla^2 \vb{N} = \nabla^2( \nabla\times \vb{M} /k) =  \nabla\times (\nabla^2\vb{M} /k) $, y por tanto la ecuación de Helmholtz para $\vb{N}$ es
	\begin{align*}
	\nabla^2 \vb{N} + k^2 \vb{N} =  \nabla\times \left( \frac{\nabla^2 \vb{M}}{k} \right) + k \nabla\times \vb{M} 
		 = \frac{1}{k} \nabla\times \left( \nabla^2 \vb{M} + k^2  \vb{M} \right).
	\end{align*}\vspace*{-1em}
	
Los campos $\vb{M}$ y $\vb{N}$ cumplen con la  ecuación de Helmholtz vectorial [Ec. \eqref{eq:Helmholtz}] si, y sólo si, la función escalar $\psi$ cumple con la ecuación de Helmholtz escalar $\nabla^2 \psi + k^2 \psi = 0$. Si este es el caso, entonces, el rotacional de $\vb{N}$ está dado por
	\begin{align}
	\nabla\times \vb{N} &= \nabla\times \qty(\frac{\nabla\times \vb{M}}{k})  
						= \frac{\nabla\qty(\nabla\cdot\vb{M})-\nabla^2\vb{M}}{k}
						= - \frac{\nabla^2 \vb{M}}{k}
						= \frac{k^2 \vb{M}}{k}
						= k \vb{M}.\label{eq:rotN}
	\end{align}\vspace*{-1em}
	
Los campos vectoriales $\vb{M}$ y $\vb{N}$ son conocidos como los \emph{armónicos  vectoriales}\index{Armónicos vectoriales}, $\psi$ como su función generadora y $\vb{r}$ como el vector de guía o vector piloto \cite{bohren1998absorption}. Los armónicos vectoriales $\vb{M}$ y $\vb{N}$  cumplen con tener divergencia nula y que el rotacional de uno es proporcional al otro [Ecs. \eqref{eq:NrotM/k} y \eqref{eq:rotN}], es decir, que cumplen con las ecuaciones de Maxwell [Ecs. \eqref{eqs:MaxwellArm}] siempre que se cumpla que\index{Armónicos vectoriales!función generadora de los}
	\begin{tcolorbox}[title = $\mathbf{\psi}$: Función generadora de los armónicos  vectoriales, ams align ]
	\nabla^2 \psi + k^2 \psi  = 0.\label{eq:AV_psi}
	\end{tcolorbox}

Cuando se considera una partícula esférica de radio $a$ e índice de refracción $n_p$, inmersa en un medio denominado matriz con índice de refracción $n_m$ (ver Fig. \ref{fig:EsferaA}), iluminada por una onda plana propagándose a lo largo del eje $z$, es conveniente emplear coordenadas esféricas $(r, \theta, \varphi)$, en las que la función generadora de los armónicos vectoriales debe cumplir con la ecuación \index{Armónicos esféricos vectoriales!función generadora de los}
	\begin{align}
	\frac{1}{r^2} \pdv{r}\qty(r^2\pdv{\psi}{r})+ 
	\frac{1}{r^2\sin\theta}\pdv{\theta}\qty(\sin\theta\pdv{\psi}{\theta})
	 + \frac{1}{r^2\sin^2\theta}\pdv[2]{\psi}{\varphi} + k^2 \psi =0. \label{eq:AEV_psi}
	\end{align}
Al resolver la Ec. \eqref{eq:AEV_psi} es posible construir un conjunto de funciones linealmente independientes que sean una base para los campos EMs incidente, esparcido y dentro de la esfera, lo que permite determinar, mediante las condiciones a la frontera de los campos EMs, la forma de la matriz de esparcimiento [Ec. \eqref{eq:MEsparcimientoGral}].

	
Para resolver la Ec. \eqref{eq:AEV_psi} se emplea el método de separación de variables, al proponer como solución $\psi= R(r)\Theta(\theta) \Phi(\varphi)$. Para que $\psi$ sea solución a la Ec.  \eqref{eq:AEV_psi}, las funciones $R(r),\, \Theta(\theta), \mbox{ y } \Phi(\varphi)$ deben cumplir con las ecuaciones
	\begin{align}
	\frac{1}{\Phi}\dv[2]{\psi}{\varphi} &+ m^2 \Phi =0, \label{eq:Phi}\\
	\frac{1}{\sin\theta}\dv{\theta}\qty(\sin\theta\dv{\Theta}{\theta}) &+ 	\qty[\ell(\ell+1)- \frac{m^2}{\sin^2\theta}]\Theta =0,\label{eq:Theta}\\
	\dv{r}\qty(r^2\dv{R}{r}) &+ \qty[ k^2 r^2 - \ell (\ell +1)] R =0, 	\label{eq:Req}
	\end{align}
en donde tanto $\ell$  como $m$ son constantes que se determinan mediante las condiciones impuestas a $\psi$. Dado que $\psi$ debe ser una función con periodicidad $2\pi$ en $\varphi$, es decir que $\psi(\varphi) = \psi(\varphi+2\pi)$, las soluciones linealmente independientes de la Ec. \eqref{eq:Phi} son \index{Armónicos esféricos vectoriales!función generadora!solución azimutal de la}

	\begin{subequations}
	\begin{align}
	\Phi_e(\varphi) &= \cos(m\varphi),\\
	\Phi_o(\varphi) = \sin(m\varphi),
	\end{align}
	\label{eq:SinCos} 
	\end{subequations} \vspace{-1em}
	
\noindent con $m$ un número natural (incluido el cero) y donde los subíndices $e$ y $o$ hacen referencia a que son funciones pares (\emph{even}, $e$) e impares (\emph{odd}, $o$), respectivamente. Las funciones $\sin(m\varphi)$ y $\cos(m\varphi)$ obedecen las relaciones de ortogonalidad
 	\begin{subequations}
	\begin{align}
	\int_0^{2\pi} \sin(m\varphi) &\cos(m' \varphi) \dd\varphi = 0 \qquad \forall\, m,m',\label{seq:ortSinCos}\\
	\int_0^{2\pi} \sin(m\varphi) \sin(m'\varphi)\dd\varphi &=  \int_0^{2\pi} \cos(m\varphi) \cos(m'\varphi)\dd\varphi  = \delta_{m,m'}\frac{\pi}{2},\label{seq:ortCos2}
	\end{align}\label{eq:ortSinCos}
 	\end{subequations}
en donde $\delta_{m,m'}$ es la delta de Kronecker.\index{Ortogonalidad!seno y coseno, relaciones de}

Al realizar el cambio de variable $\mu = \cos\theta$ en la Ec. \eqref{eq:Theta}, ésta se reescribe como
	\begin{align*}
	\qty(1-\mu^2) \dv[2]{\Theta}{\mu} - 2 \mu \dv{\Theta}{\mu} + \qty[\ell(\ell+1)-\frac{m^2}{(1-\mu^2)}]\Theta= 0,
	\end{align*}\index{Armónicos esféricos vectoriales!función generadora!solución polar de la}\index{Ecuación!asociada de Legendre}\index{Legendre!ecuación asociada de}
\hspace{-.5em}cuyas soluciones son	las \emph{funciones asociadas de Legendre} $P_\ell^m(\cos\theta)$ de grado $\ell$ y orden $m$  \cite{arfken2001methods}, imponiendo que $\ell = m, m+1,m+2,\ldots$ para  que la Ec. \eqref{eq:Theta} sea finita en $\theta = 0$ y $\theta = \pi$ ---o bien $\mu=\pm1$---. Las funciones asociadas de Legendre cumplen con la relación de ortogonalidad 
	\begin{align}
	\int_{-1}^1P_\ell^m(\mu) P_{\ell'}^md\mu = \delta_{\ell,\ell'}\frac{2}{2\ell+1}\frac{(\ell+m)!}{(\ell-m)!}.
	\label{eq:ortLegendre}
	\end{align}\index{Legendre!polinomios de}\index{Legendre!funciones asociadas de}\index{Ortogonalidad!funciones asociadas de Legrende, relaciones de}\index{Legendre!funciones asociadas de!relaciones de ortogonalidad de las}
\hspace{-.5em}Asimismo, las funciones asociadas de Legendre se reducen a los polinomios de Legendre cuando $m=0$, además de que las funciones asociadas y los polinomios de Legendre se relacionan mediante la identidad  \cite{arfken2001methods}
	\begin{align}
	P_\ell^m (\mu) = (1-\mu^2)^{m/2}\dv[m]{P_\ell(\mu)}{\mu},
	\label{eq:Legendre}
	\end{align}
de donde se deduce  que $P_\ell^m(\pm 1)=0$ para toda $m$ distinta de cero.

Para resolver la Ec. \eqref{eq:Req} se emplea el cambio de variable $\rho = k r$ y de define la función $Z =R\sqrt{\rho}$, por lo que la ecuación radial se reescribe como \index{Armónicos esféricos vectoriales!función generadora!solución radial de la}\index{Ecuación!esférica de Bessel}\index{Bessel!ecuación esférica de}
	\begin{align}
	\rho \dv{\rho}\qty(\rho\dv{Z}{\rho})+\qty[\rho^2-\qty(\ell+\frac12)^2] Z = 0,
	\label{eq:rho}
	\end{align}
cuyas soluciones son las \emph{funciones esféricas de Bessel} $j_\ell$ y $y_\ell$ o cualquier combinación lineal de ellas, por lo que de forma general las soluciones de la Ec. \eqref{eq:rho} son \cite{arfken2001methods} \index{Bessel!funciones esféricas de}

	\begin{subequations}
	\begin{align}
	j_\ell (\rho) &= \sqrt{\frac{\pi}{2\rho}} J_{\ell+1/2}(\rho), \label{eqs:jn}\\
	y_\ell (\rho) &= \sqrt{\frac{\pi}{2\rho}} Y_{\ell+1/2}(\rho), \label{eqs:yn}\\
	h_\ell^{(1)} (\rho) &= j_\ell(\rho) + i y_\ell(\rho), \label{eqs:h1}\\
	h_\ell^{(2)} (\rho) &=  j_\ell(\rho) - i y_\ell(\rho), \label{eqs:h2}
	\end{align}			\label{eq:SphBessel}
	\end{subequations}

\noindent	
en donde $J_\ell$ y $Y_\ell$ son las \emph{funciones de Bessel del primer y segundo tipo}, respectivamente, y $h_\ell$ son las \emph{funciones esféricas de Bessel del tercer tipo}, también denominadas como \emph{funciones esféricas de Hankel}. Todas las funciones esféricas de Bessel $z_\ell$ ---donde $z_\ell$ es cualquier función de las Ecs. \eqref{eq:SphBessel}--- puede ser calculada mediante relaciones de recurrencia\footnote{Todas las funciones esféricas de Bessel cumplen: $	z_{\ell-1}(\rho) + z_{\ell+1}(\rho) =(2\ell+1)z_\ell(\rho)/\rho$ y $(2\ell + 1) \dv*{z_\ell(\rho)}{\rho} = \ell z_{\ell-1}(\rho) - (\ell+1)z_{\ell+1}(\rho)$, con  $j_0(\rho) = \sin\rho / \rho$ y $j_1(\rho) = \sin\rho / \rho^2- \cos\rho/\rho$, $y_0(\rho) = -\cos\rho/\rho$ y $y_1(\rho) = -\cos\rho/\rho^2-\sin\rho/\rho$.\index{Bessel!funciones esféricas de!relaciones de recurrencia de las}} \cite{arfken2001methods}\index{Hankel!funciones esféricas de}\index{Hankel|see{Bessel}}.

Dado que las soluciones para la ecuación azimutal son las Ecs. \eqref{eq:SinCos}, para la polar, Ec. \eqref{eq:Legendre} y para la radial, Ecs. \eqref{eq:SphBessel}, las funciones generadoras de los armónicos esféricos vectoriales son\index{Armónicos esféricos vectoriales!función generadora!solución general}\begin{subequations}\vspace*{-2em}

	\begin{align}
	\psi_{em\ell} = \cos(m\varphi) P_\ell^m( \cos \theta) z_\ell(k r),\\
	\psi_{om\ell} = \sin(m\varphi) P_\ell^m( \cos \theta) z_\ell(k r).
	\end{align}
	\label{eq:psieo}	\end{subequations}\vspace*{-1em}		

\noindent Al emplear las Ecs. \eqref{eq:psieo} en la Ec. \eqref{eq:MrotCPsi} se obtiene como resultado $\vb{M}_{em\ell}$ y $\vb{M}_{om\ell}$, dados por las expresiones 

	\begin{subequations}
	\begin{tcolorbox}[title = Armónicos esféricos vectoriales $\vb{M}_{em\ell}$ y $\vb{M}_{om\ell}$, ams align ]
	\vb{M}_{em\ell} = &-m\sin(m\varphi)z_\ell(kr) \frac{P_\ell^m(\cos\theta)}{\sin\theta}\,\vu{e}_\theta
					-\cos(m\theta)z_\ell(kr) \dv{P_\ell^m(\cos\theta)}{\theta}(\cos\theta)\,\vu{e}_\varphi,\label{seq:Meml} \\
	\vb{M}_{om\ell} = & m\cos(m\varphi)z_\ell(kr) \frac{P_\ell^m(\cos\theta)}{\sin\theta}\,\vu{e}_\theta
					-\sin(m\theta)z_\ell(kr) \dv{P_\ell^m(\cos\theta)}{\theta}(\cos\theta)\,\vu{e}_\varphi.	\label{seq:Moml}				
	\end{tcolorbox}  \noindent
%
Para el cálculo $\vb{N}_{em\ell}$ y $\vb{N}_{om\ell}$ se sustituyen las Ecs. \eqref{seq:Meml} y \eqref{seq:Moml} en la Ec. \eqref{eq:NrotM/k}. Para simplificar las expresiones de las componentes radiales de  $\vb{N}_{em\ell}$ y $\vb{N}_{om\ell}$, se agrupan los términos que dependen de $\varphi$ y $kr$ y, dado que las funciones asociadas de Legendre cumplen con la relación 
\begin{align*}
-\ell(\ell+1) P_\ell^m (\cos\theta)= \frac{1}{\sin\theta}\dv{\theta}\qty(\sin\theta\dv{P_\ell^m(\cos\theta)}{\theta}) - \frac{m^2}{\sin^2\theta}P_\ell^m(\cos\theta),
\end{align*}
que es una consecuencia de la Ec. \eqref{eq:Theta}, las expresiones de $\vb{N}_{em\ell}$ y $\vb{N}_{om\ell}$ son \index{Armónicos esféricos vectoriales!$\vb{M}$ y $\vb{N}$} 
%
	\begin{tcolorbox}[title = Armónicos esféricos vectoriales $\vb{N}_{em\ell}$ y $\vb{N}_{om\ell}$, ams align, breakable ]
	\vb{N}_{em\ell} =&\cos(m\varphi) \frac{z_\ell(kr)}{kr} \ell(\ell+1)P_\ell^m(\cos\theta)\,\vu{e}_r\notag\\
	&+ \cos(m\varphi)  \frac{1}{kr} \dv{(kr)}\qty\Big[kr\, z_\ell(kr)] \dv{P_\ell^m(\cos\theta)}{\theta}(\cos\theta)\,\vu{e}_\theta
	 \label{seq:Neml} \\
		&- m \sin(m\varphi) \frac{1}{kr} \dv{(kr)}\qty\Big[kr\, z_\ell(kr)] \frac{P_\ell^m(\cos\theta)}{\sin\theta}
		 \,\vu{e}_\varphi, \notag\\			
	\vb{N}_{om\ell} =&\sin(m\varphi)\frac{z_\ell(kr)}{kr} \ell(\ell+1)P_\ell^m(\cos\theta)\,\vu{e}_r \notag\\
	&+ \sin(m\varphi)  \frac{1}{kr} \dv{(kr)}\qty\Big[kr\, z_\ell(kr)] \dv{P_\ell^m(\cos\theta)}{\theta}(\cos\theta) \,\vu{e}_\theta
	 \label{seq:Noml} \\
		&+ m \cos(m\varphi) \frac{1}{kr} \dv{(kr)}\qty\Big[kr\, z_\ell(kr)] \frac{P_\ell^m(\cos\theta)}{\sin\theta}
		\, \vu{e}_\varphi. \notag							
	\end{tcolorbox}\label{eq:AEV}
	\end{subequations}

Los armónicos esféricos vectoriales son solución a la ecuación de Helmholtz, por lo que cualquier solución de los campos EMs puede escribirse como una serie infinta en términos de las Ecs. \eqref{eq:AEV}. Para resolver el problema de los campos EMs esparcidos por una partícula esférica, esto es, determinar las componentes de la matriz de esparcimiento $\mathbb{S}$ de la Ec. \eqref{eq:MEsparcimientoGral}, se expande una onda plana $\vb{E}^i$ en la base de los armónicos esféricos vectoriales, haciendo uso de sus condiciones de ortogonalidad, calculadas a partir de la relaciones de ortogonalidad de las Ecs. \eqref{eq:ortSinCos} y \eqref{eq:ortLegendre}, dando como resultado que los armónicos esféricos vectoriales son ortogonales cuando tienen paridad distinta y cuando se realiza el producto interior entre $\vb{M}$ y $\vb{N}$, es decir \index{Armónicos esféricos vectoriales!relaciones de ortogonalidad de los}\index{Ortogonalidad!armónicos esféricos vectoriales, relaciones de}
%
	\begin{tcolorbox}[ ams align ]
		\langle\vb{M}_{em\ell}, \vb{M}_{om'\ell'} \rangle_{\theta,\varphi} =
		\langle\vb{N}_{em\ell}, \vb{N}_{om'\ell'} \rangle_{\theta,\varphi} = 0
		&\qquad \forall\,  m,m',\ell, \ell',\\
		\langle\vb{M}_{om\ell}, \vb{N}_{em'\ell'} \rangle_{\theta,\varphi} = 
		\langle\vb{M}_{om\ell}, \vb{N}_{om'\ell'} \rangle_{\theta,\varphi} = 	
		\langle\vb{M}_{em\ell}, \vb{N}_{em'\ell'} \rangle_{\theta,\varphi} = 0
		&\qquad \forall\,  m,m',\ell, \ell',	\\
		\langle\vb{M}_{em\ell},  \vb{N}_{om\ell'} \rangle_{\theta,\varphi} =
		\langle\vb{M}_{om\ell},  \vb{N}_{em\ell'} \rangle_{\theta,\varphi} = 0	
		&\qquad \forall\, \ell, \ell'\, m,
	\end{tcolorbox}\noindent
en donde se definió el producto interior $\langle \vb{A},\vb{B} \rangle_{\theta,\varphi}$ como 
	\begin{align*}
	\langle \vb{A},\vb{B} \rangle_{\theta,\varphi} 
	\equiv 
	\int_0^{2\pi}\int_0^\pi \vb{A}\cdot\vb{B} \sin\theta \dd\theta \dd\varphi.
	\end{align*}
De igual manera, cuando se realiza el producto interior con elementos de los armónicos esféricos vectoriales de la misma paridad, y considerando las combinaciones de  $\langle \vb{M},\vb{M}\rangle_{\theta,\varphi}$ y $\langle \vb{N},\vb{N}\rangle_{\theta,\varphi}$  se obtienen las relaciones \vspace{-.5em}
	\begin{tcolorbox}[ ams align ]
	\!\!	\langle\vb{M}_{em\ell},  \vb{M}_{em\ell'} \rangle_{\theta,\varphi} = 
		&\langle\vb{M}_{om\ell},  \vb{M}_{om\ell'} \rangle_{\theta,\varphi} 
			=\delta_{\ell,\ell'}\pi z_\ell (\rho)^2
			\frac{\ell(\ell+1)}{2\ell+1}\frac{(\ell+m)!}{(\ell-m)!}
		\quad \forall\, \ell, \ell',\, m, \label{eq:MM} \\
	\!\!	\langle\vb{N}_{em\ell},  \vb{N}_{em\ell'} \rangle_{\theta,\varphi} = 
		&\langle\vb{N}_{om\ell},  \vb{N}_{om\ell'} \rangle_{\theta,\varphi}
		 \label{eq:NN}\\
			=&\delta_{\ell,\ell'}\pi\frac{\ell(\ell+1)}{2\ell+1}
			\frac{(\ell+m)!}{(\ell-m)!}
			\left\{ \qty[\frac{z_\ell(\rho)}{\rho}]^2 \ell(\ell+1)+
			 \qty[\frac{1}{\rho}\dv{[\rho z_\ell (\rho)]}{\rho}]^2  \right\}
		\quad \forall\, \ell, \ell',\, m.	\notag
	\end{tcolorbox}


	Sea una onda plana con longitud de onda $\lambda$, polarizada en la dirección $x$, y caracterizada por el campo eléctrico $\va{E}^i$ propagándose en la dirección $z$ en un medio con índice de refracción $n_m$. En la base de los vectores ortonormales polares la onda plana se escribe como
	\begin{align}
	\va{E}^i =& E_0 e^{ikz}\vu{e}_x \notag\\
			 =& E_0 e^{ikr\cos\theta} \qty(\sin\theta\cos\varphi \vu{e}_r + 
	\cos\theta\cos\varphi\vu{e}_\theta-\sin\varphi\vu{e}_\phi),\label{eq:OndaPlanaPolares}
	\end{align}
donde $E_0$ es la magnitud del campo eléctrico y $k=2\pi n_m/\lambda$. La expansión de la ec. \eqref{eq:OndaPlanaPolares} en términos de los armónicos esféricos vectoriales [ecs. \eqref{eq:AEV}] es
	\begin{equation}
	\va{E}^i = \sum_{m=0}^\infty\sum_{\ell=m}^\infty
			\qty[B_{em\ell}\va{M}_{em\ell} + B_{om\ell}\va{M}_{om\ell} +
				A_{em\ell}\va{N}_{em\ell} + A_{om\ell}\va{N}_{om\ell}].\label{eq:OndaPlanaAEV}
	\end{equation}
Para determinar los coeficientes $B_{em\ell},\, B_{om\ell},\, A_{em\ell}$ y $ A_{om\ell}$ se requieren las relaciones de ortogonalidad de los armónicos esféricos vectoriales.\\

 Las funciones $\sin(m\varphi)$ y $\cos(m\varphi)$ obedecen las relaciones de ortogonalidad
 	\begin{subequations}
	\begin{align}
	\int_0^{2\pi} \sin(m\varphi) &\cos(m' \varphi) \dd\varphi = 0 \qquad \forall\, m,m',\label{seq:ortSinCos}\\
	\int_0^{2\pi} \sin(m\varphi) \sin(m'\varphi)\dd\varphi &=  \int_0^{2\pi} \cos(m\varphi) \cos(m'\varphi)\dd\varphi  = \delta_{m,m'}\frac{\pi}{2}.\label{seq:ortCos2}
	\end{align}\label{eq:SinCos}
 	\end{subequations}
Por la ec. \eqref{seq:ortSinCos} se cumple que el producto interior\footnote{Se define el producto interior $\langle \va{A},\va{B} \rangle_{\theta,\varphi}$ como $\langle \va{A},\va{B} \rangle_{\theta,\varphi} \equiv \int_0^{2\pi}\int_0^\pi \va{A}\cdot\va{B} \sin\theta \dd\theta \dd\varphi$} entre $\va{M}_{em\ell}$ y $\va{M}_{om'\ell'}$, y $\va{N}_{em\ell}$ y $\va{N}_{om'\ell'}$ es
	\begin{tcolorbox}[ ams align ]
		\langle\va{M}_{em\ell}, \va{M}_{om'\ell'} \rangle_{\theta,\varphi} =
		\langle\va{N}_{em\ell}, \va{N}_{om'\ell'} \rangle_{\theta,\varphi} = 0
		&\qquad \forall\,  m,m',\ell, \ell',\\
		\intertext{así como también}
		\langle\va{M}_{om\ell}, \va{N}_{em'\ell'} \rangle_{\theta,\varphi} = 
		\langle\va{M}_{om\ell}, \va{N}_{om'\ell'} \rangle_{\theta,\varphi} = 	
		\langle\va{M}_{em\ell}, \va{N}_{em'\ell'} \rangle_{\theta,\varphi} = 	0
		&\qquad \forall\,  m,m',\ell, \ell'.				
	\end{tcolorbox}\noindent
pues $\va{M}$ tiene componente nula en $\vu{e}_r$ y en los demás términos se encuentra la ec. \eqref{seq:ortSinCos}. Las ecs. \eqref{eq:SinCos} implican que todos los armónicos esféricos vectoriales orden $m$ distinto  son ortogonales entre sí.\\

 El producto interior entre $\va{M}_{em\ell}$ y $\va{N}_{om\ell'}$, empleando el resultado de la ec. \eqref{seq:ortCos2} con $m=m'$, está dado por
	\begin{align}
		\langle\va{M}_{em\ell},  \va{N}_{om\ell'} \rangle_{\theta,\varphi} &= - \frac{\pi}{2} \frac{z_\ell (\rho)}{\rho}\dv{z_{\ell'}(\rho)}{\rho}	m
							 \int_0^\pi\qty[P_\ell^m(\cos \theta )\dv{P_{\ell'}^m(\cos \theta )}{\theta}+ 
							  \dv{P_{\ell}^m(\cos \theta )}{\theta}P_{\ell'}^m(\cos \theta )] \dd\theta \notag\\
					 &=- \frac{\pi}{2} \frac{z_\ell (\rho)}{\rho}\dv{z_{\ell'}(\rho)}{\rho}	m 
					 	\int_0^\pi \dv{\theta}\qty[P_\ell^m(\cos\theta)P_{\ell'}^m(\cos\theta)]\dd\theta \notag\\
					 &=- \frac{\pi}{2} \frac{z_\ell (\rho)}{\rho}\dv{z_{\ell'}(\rho)}{\rho}	m \eval{P_\ell^m(\cos\theta)P_{\ell'}^m(\cos\theta)}_0^\pi.
					 	\label{eq:MeNo}
	\end{align}
Mediante un procedimiento semejante se obtiene que $\langle\va{M}_{em\ell},  \va{N}_{om\ell'} \rangle_{\theta,\varphi}=\langle\va{M}_{om\ell},  \va{N}_{em\ell'} \rangle_{\theta,\varphi}$. Haciendo uso de la relación entre las funciones asociadas de Legndre con los polinomios de Legendre [ec. \eqref{eq:FAL-PL}] se obtiene que $P_\ell^m(\cos\theta)=0$ para $\theta=0,\pi$ y $m\neq 0$. Sin embargo, si en la ec. \eqref{eq:MeNo} $m$ es igual a cero, el producto interior también es nulo, por lo que se cumple que 
	\begin{tcolorbox}[ ams align ]
		\langle\va{M}_{em\ell},  \va{N}_{om\ell'} \rangle_{\theta,\varphi}=
		\langle\va{M}_{om\ell},  \va{N}_{em\ell'} \rangle_{\theta,\varphi}= 0	
		\qquad \forall\, \ell, \ell'\, m.
	\end{tcolorbox}
Las expresiones del  producto interior entre $\va{M}_{em\ell}$ y $\va{M}_{em\ell'}$, y $\va{N}_{em\ell}$ y $\va{N}_{em\ell'}$, empleando el resultado de la ec. \eqref{seq:ortCos2} con $m=m'$, y la relación de ortogonalidad de las funciones asociadas de Legendre [ec. \eqref{eq:ortLegendre}] son
	\begin{align*}
		\langle\va{M}_{em\ell},  \va{M}_{em\ell'} \rangle_{\theta,\varphi} =& 
				\frac{\pi}{2} z_\ell (\rho) z_{\ell'}(\rho)\times	 \int_0^\pi\qty[\frac{m^2}{\sin^2\theta}P_\ell^m(\cos\theta)P_{\ell'}^m(\cos\theta)
				 +\dv{P_\ell^m(\cos\theta)}{\theta}\dv{P_{\ell'}^m(\cos\theta)}{\theta}]\sin\theta\dd\theta\\
		\langle\va{N}_{em\ell},  \va{N}_{em\ell'} \rangle_{\theta,\varphi} =&
				\frac{\pi}{2} \qty[\frac{z_\ell(\rho)}{\rho}\ell(\ell+1)]^2\frac{2}{2\ell+1}\frac{(\ell+m)!}{(\ell-m)!}\delta_\ell^{\ell'}+
				\frac{\pi}{2} \frac{1}{\rho^2}\dv{\rho}\qty[\rho z_\ell (\rho)]\dv{\rho}\qty[\rho z_{\ell'}(\rho)]\\
				&\times \int_0^\pi\qty[\frac{m^2}{\sin^2\theta}P_\ell^m(\cos\theta)P_{\ell'}^m(\cos\theta)
					+\dv{P_\ell^m(\cos\theta)}{\theta}\dv{P_{\ell'}^m(\cos\theta)}{\theta}]\sin\theta\dd\theta.				 
	\end{align*}
Asimismo, se cumple que  $\langle\va{M}_{em\ell},  \va{M}_{em\ell'} \rangle_{\theta,\varphi}=\langle\va{M}_{om\ell},  \va{M}_{om\ell'} \rangle_{\theta,\varphi}$ y $\langle\va{N}_{em\ell},  \va{N}_{em\ell'} \rangle_{\theta,\varphi}=\langle\va{N}_{om\ell},  \va{N}_{om\ell'} \rangle_{\theta,\varphi}$. Sustituyendo $P_\ell^m(\cos\theta)$ en la ec. \eqref{eq:Theta} y multiplicándola por $P_{\ell'}^m(\cos\theta)$, operando de la misma forma intercambiando los papeles de $P_\ell^m(\cos\theta)$  y $P_{\ell'}^m(\cos\theta)$ y sumando ambos resultados se llega a la expresión 
	\begin{align}
	2\frac{m^2}{\sin^2\theta}P_\ell^m P_{\ell'}^m \sin\theta =&
					 P_\ell\dv{\theta}\qty[\sin\theta \dv{P_{\ell'}^m}{\theta}] P_{\ell'}\dv{\theta}\qty[\sin\theta \dv{P_\ell^m}{\theta}]
					 +	\ell(\ell+1)P_\ell^m P_{\ell'}^m	 \sin\theta  \notag\\
					 &+ \ell'(\ell'+1)P_\ell^m P_{\ell'}^m\sin\theta, \label{eq:PnPn'}
	\end{align}
en donde se obvia el argumento $\cos\theta$. Dado que 
	\begin{equation*}
	\dv{\theta}\qty[P_{\ell'}^m\sin\theta  \dv{P_\ell^m}{\theta}] 
	= P_{\ell'}^m\dv{\theta}\qty[\sin\theta \dv{P_\ell^m}{\theta}] + \sin\theta\dv{P_{\ell'}^m}{\theta}\dv{P_\ell^m}{\theta},
	\end{equation*}
sumando $2\sin\theta\dd P_{\ell'}^m\dd\theta \dd P_\ell^m \dd\theta$ de ambos lados de la ec. \eqref{eq:PnPn'} y agrupando términos, se obtiene que el integrando presente en los productos interiores de  $\va{M}_{em\ell}$ y $\va{M}_{em\ell'}$, y $\va{N}_{em\ell}$ y $\va{N}_{em\ell'}$ es
	\begin{align*}
	\qty[\frac{m^2}{\sin^2\theta}P_\ell^mP_{\ell'}^m	+\dv{P_\ell^m}{\theta}\dv{P_{\ell'}^m}{\theta}]\sin\theta = &
					 \frac12 \dv{\theta}\qty[\sin\theta \dv{P_{\ell'}^m}{\theta}P_\ell^m + \sin\theta\dv{P_{\ell}^m}{\theta}P_{\ell'}^m ]
					 +	\frac12 \ell(\ell+1)P_\ell^m P_{\ell'}^m	 \sin\theta  \\
					 &+ \frac12 \ell'(\ell'+1)P_\ell^m P_{\ell'}^m\sin\theta,
	\end{align*}
en donde el primer término de la suma se desvanece al evaluarse en $\theta=0,\pi$ y los últimos cumplen con la relación de ortogonalidad de la ec. \eqref{eq:ortLegendre}. Por lo tanto
	\begin{tcolorbox}[ ams align ]
		\langle\va{M}_{em\ell},  \va{M}_{em\ell'} \rangle_{\theta,\varphi}=&	\langle\va{M}_{om\ell},  \va{M}_{om\ell'} \rangle_{\theta,\varphi} 
										=\delta_\ell^{\ell'}\pi z_\ell (\rho)^2\frac{\ell(\ell+1)}{2\ell+1}\frac{(\ell+m)!}{(\ell-m)!}
		&\qquad \forall\, \ell, \ell',\, m, \label{eq:MM} \\
		\langle\va{N}_{em\ell},  \va{N}_{em\ell'} \rangle_{\theta,\varphi}=&	\langle\va{N}_{om\ell},  \va{N}_{om\ell'} \rangle_{\theta,\varphi} \notag\\
										=&\delta_\ell^{\ell'}\pi\frac{\ell(\ell+1)}{2\ell+1}\frac{(\ell+m)!}{(\ell-m)!}\left\{ \qty[\frac{z_\ell(\rho)}{\rho}]^2 \ell(\ell+1)+
										   \qty[\frac{1}{\rho}\dv{[\rho z_\ell (\rho)]}{\rho}]^2  \right\}
		&\qquad \forall\, \ell, \ell',\, m.	\label{eq:NN}
	\end{tcolorbox}

Con las expresiones de los armónicos esféricos vectoriales  [ec. \eqref{eq:AEV}] y con sus relaciones de ortogonalidad se pueden calcular los coeficientes de la expansión de una onda plana en esta base [ec. \eqref{eq:OndaPlanaAEV}]. Igualando las ecs. \eqref{eq:OndaPlanaPolares} y \eqref{eq:OndaPlanaAEV}, se obtiene
	\begin{align*}
\va{E}^i = & E_0 e^{ikr\cos\theta} \qty(\sin\theta\cos\varphi \vu{e}_r + 
								\cos\theta\cos\varphi\vu{e}_\theta-\sin\varphi\vu{e}_\varphi)\\
	 =& \sum_{m=0}^\infty\sum_{\ell=m}^\infty \qty[ B_{em\ell}\va{M}_{em\ell} 
	 	+ B_{om\ell}\va{M}_{om\ell} +A_{em\ell}\va{N}_{em\ell} + A_{om\ell}\va{N}_{om\ell}].
	\end{align*}
Dado que en la componente radial hay dependencia con $\cos\varphi$, se sigue que $m=1$ al comparar con las expresiones de $\va{N}_{em\ell}$ [ec. \eqref{seq:Neml}] y $\va{N}_{om\ell}$ [ec. \eqref{seq:Noml}] ---únicos elementos con componente radial---, y además que $A_{om\ell}=0$ pues $\va{N}_{om\ell}$ es proporcional a $\sin\varphi$ en esta componente. Asimismo, por la dependencia con $\sin\varphi$ en la componente  $\vu{e}_\varphi$, $B_{em\ell}=0$ pues $\va{M}_{em\ell}$ es proporcional a $\cos\varphi$ en dicha entrada. \\

La onda plana no tiene ninguna divergencia, por lo que se escoge $z_\ell = j_\ell$. Esto se denota en los armónicos esféricos vectoriales con el superíndice (1). Se escribe entonces a la onda plana como
	\begin{equation}
	\va{E}^i = \sum_{\ell=1}^\infty \qty[B_{o1\ell}\va{M}_{o1\ell}^{(1)} + A_{e1\ell}\va{N}_{e1\ell}^{(1)}],
	\end{equation}
con\\

	\begin{align}
	B_{o1\ell} = \frac{\langle \va{E}^i, \va{M}_{o1\ell}^{(1)}  \rangle_{\theta,\varphi}}
						{\langle \va{M}_{o1\ell}^{(1)} ,\va{M}_{o1\ell}^{(1)} \rangle}_{\theta,\varphi},\notag\\
	A_{e1\ell} = \frac{\langle \va{E}^i, \va{N}_{e1\ell}^{(1)} \rangle_{\theta,\varphi}}
						{\langle \va{N}_{e1\ell}^{(1)},\va{N}_{e1\ell}^{(1)} \rangle}_{\theta,\varphi}						.
	\end{align}

Empleando la ec. \eqref{eq:MM} con $m=1$, se calcula el denominador del coeficiente $B_{o1\ell}$, como
	\begin{equation}
			\langle \va{M}_{o1\ell}^{(1)} ,\va{M}_{o1\ell}^{(1)} \rangle_{\theta,\varphi} = \pi \frac{[j_\ell(\rho) \ell(\ell+1)]^2}{2\ell+1},
			\label{eq:MMm=1}
	\end{equation}
y con la ec. \eqref{seq:Moml} y la ec. \eqref{eq:OndaPlanaPolares} se calcula
	\begin{align}
		\langle \va{E}^i, \va{M}_{o1\ell}^{(1)}  \rangle_{\theta,\varphi} 
			=& \frac{E_0 \pi}{2} j_\ell(\rho) \int_0^\pi \qty(\cos\theta P_\ell^1(\cos\theta)+\sin\theta \dv{P_\ell^1(\cos\theta)}{\theta})e^{i\rho\cos\theta}\dd\theta,\notag\\
			=& \frac{E_0 \pi}{2} j_\ell(\rho) \int_0^\pi \dv{\theta}\qty[\sin\theta P_\ell^1(\cos\theta)]e^{i\rho\cos\theta}\dd\theta. \label{eq:EiM_1}
	\end{align}
Considerando la relación entre las funciones asociadas de Legendre y los polinomios de Legendre [ec. \eqref{eq:FAL-PL}] con $m=1$, se cumple que $P_\ell^1(\mu)=-\dd P_\ell(\mu)/\dd\theta$. Además, los polinomios de Legedre cumplen con la ec. \eqref{eq:Theta} con $m=0$, es decir,
	\begin{equation*}
	\dv{\theta}\qty[\sin\theta \dv{P_\ell(\cos\theta)}{\theta}] = - \ell(\ell+1)P_\ell(\cos\theta) \sin\theta,
	\end{equation*}
por lo que la ec. \eqref{eq:EiM_1} es
	\begin{align}
	\langle \va{E}^i, \va{M}_{o1\ell}^{(1)}  \rangle_{\theta,\varphi}
			 = & \frac{E_0 \pi}{2} j_\ell(\rho) \int_0^\pi -\dv{\theta}\qty[\sin\theta \dv{P_\ell(\cos\theta)}{\theta}]e^{i\rho\cos\theta}\dd\theta\notag\\
			  = & \frac{E_0 \pi}{2} j_\ell(\rho) \ell(\ell+1)\int_0^\pi P_\ell(\cos\theta)\sin\theta e^{i\rho\cos\theta}\dd\theta.\label{eq:EiM_2}
	\end{align}
La generalización de las integrales de Poisson de Gegenbauer  relaciona a los polinomios de Legendre con la función esférica de Bessel de primer tipo mediante la relación
	\begin{equation*}
	j_\ell(\rho) = \frac{1}{2i^\ell}\int_0^\pi P_\ell(\cos\theta)\sin\theta e^{i\rho\cos\theta}\dd\theta,
	\end{equation*}
por lo que la ec. \eqref{eq:EiM_2} se reescribe como
	\begin{equation}
	\langle \va{E}^i, \va{M}_{o1\ell}^{(1)}  \rangle_{\theta,\varphi}
			=  E_0 \pi [j_\ell(\rho)]^2 \ell(\ell+1) i^\ell.\label{eq:EiM}
	\end{equation}
Calculando el cociente de la ec.  \eqref{eq:EiM} con \eqref{eq:MMm=1}, se calcula el coeficiente $B_{o1\ell}$, dado por la expresión
	\begin{equation}
	B_{o1\ell} = i^\ell  E_0 \frac{ (2\ell+1)}{\ell(\ell+1)}.
	\end{equation}

El denominador del coeficiente, empleando la ec. \eqref{eq:NN}, es
	\begin{equation}
	\langle \va{N}_{e1\ell}^{(1)},\va{N}_{e1\ell}^{(1)} \rangle_{\theta,\varphi} 
				= \pi \frac{[\ell(\ell+1)]^2}{2\ell+1} 
				\left\{ \qty[\frac{z_\ell(\rho)}{\rho}]^2 \ell(\ell+1)+  \qty[\frac{1}{\rho}\dv{[\rho z_\ell (\rho)]}{\rho}]^2  \right\}
	\end{equation}

Al emplear las Ecs. \eqref{eq:MM} y \eqref{eq:NN} con $m=1$, y las condiciones de ortogonalidad de los armónicos esféricos vectoriales, se calcula la expresión de la onda plana en una base esférica, dada por
	\begin{subequations}
	\begin{equation}
	\vb{E}^i = E_0 \sum_{\ell =1}^\infty i^\ell \frac{2\ell+1}{\ell(\ell+1)}\qty(\vb{M}_{o1\ell}^{(1)}-i\vb{N}_{e1\ell}^{(1)}).
	\label{eqs:EiAEV}
	\end{equation}
El campo magnético incidente se calcula empleando la Ley de Farady-Lenz [Ec. \eqref{seq:FLArm}], cuyo resultado es
	\begin{equation}
	\vb{H}^i =\frac{-k_m}{\omega\mu_m} \sum_{\ell =1}^\infty  E_\ell\qty(\vb{M}_{e1\ell}^{(1)}+i\vb{N}_{o1\ell}^{(1)}),
	\label{eqs:HiAEV}
	\end{equation}\label{eq:EHiAEV}
	\end{subequations}
con $E_\ell = E_0 i^\ell (2\ell+1)/[\ell(\ell+1)]$.