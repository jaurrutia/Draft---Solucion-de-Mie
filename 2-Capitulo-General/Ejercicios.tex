%!TeX root = ../main.tex

\section{Ejercicios sugeridos}

\begin{enumerate}[label=\textbf{Ejercicio \thechapter.\arabic*},resume]
\item A partir de las Ecs. \eqref{eq:H=curlA} y \eqref{eq:Aoscilante}, muestre que el campo $\vb{H}$, para una fuente oscilante en el tiempo, está dado por la Ec. \eqref{eq:Hoscilante}.
\label{ex:Hoscilante}
\item[\color{blue} Solución:]
	Notemos que
	\begin{align*}
	\nabla\times\qty(\frac{e^{ikr}}{r}\vb{p}) = \nabla\qty(\frac{e^{ikr}}{r})\times\vb{p} + \frac{e^{ikr}}{r}\nabla\times\vb{p},
\end{align*}		
y como $\vb{p}$ no depende de $\vb{r}$, entonces $\nabla\times\vb{p} = \vb{0}$. Por lo tanto
\begin{align*}
\nabla\times\qty(\frac{e^{ikr}}{r}\vb{p}) 
	 &= \qty[e^{ikr}\nabla\qty(\frac{1}{r})+\frac{1}{r}\nabla\qty(e^{ikr})]\times\vb{p}\\
	 &= \qty[e^{ikr}\qty(-\frac{\vu{e}_r}{r^2})+\frac{ik}{r}e^{ikr}] \times\vb{p}\\
	 &= \frac{e^{ikr}}{r}\qty(-\frac{1}{r}+ik)\vu{e}_r\times\vb{p}.
\end{align*}
Multiplicando el resultado anterior por $-i\omega k/4\pi k$ y recordando que la relación de dispersión para ondas electromagnéticas en el vacío es $\omega =  ck$,  obtenemos que
	\begin{align*}
	\vb{H} = \frac{ck^2}{4\pi}\frac{e^{ikr}}{r}\qty(1-\frac{1}{ikr})\vu{e}_r\times\vb{p}.
	\end{align*}
	
\item Muestre que el campo eléctrico para una fuente oscilante en el tiempo está dada por la Ec. \eqref{eq:Eoscilante}. Hágalo calculando el rotacional de la Ec.  \eqref{eq:Hoscilante}.
\label{ex:Eoscilante}
\item[\color{blue} Solución:]
	Calculando el rotacional de la Ec. \eqref{eq:Hoscilante} como el producto de una dunciín escalar que depende de $r$ y el término vectorial $\vu{e}_r\times\vb{p}$, se obtiene que el campo eléctrico es
	%
	\begin{align*}
	\vb{E} &= \frac{i}{k}\sqrt{\frac{\mu_0}{\varepsilon_0}} \frac{ck^2}{4\pi}\left\{
		\qty[\frac{e^{ikr}}{r}\qty(1-\frac{1}{ikr})]\nabla\times(\vu{e}_r\times\vb{p}) +
	\nabla\qty[\frac{e^{ikr}}{r}\qty(1-\frac{1}{ikr})]\times(\vu{e}_r\times\vb{p}) 
	\right\}\\
		&= \frac{ik}{4\pi\varepsilon_0}\left\{
		\qty[\frac{e^{ikr}}{r}\qty(1-\frac{1}{ikr})]\nabla\times(\vu{e}_r\times\vb{p}) +
	\dv{r}\qty[\frac{e^{ikr}}{r}\qty(1-\frac{1}{ikr})]\vu{e}_r\times(\vu{e}_r\times\vb{p}) 
	\right\}.
	\end{align*}
%
Al descomponer el momento dipolar $\vb{p}$ en una componente longitudinal y una transversal como $\vb{p} = (\vb{p}\cdot\vu{e}_r)\vu{e}_r + (\vb{p}\times\vu{e}_r)\times\vu{e}_r $, se obtiene que
%
\begin{align*}
	\vu{e}_r \times (\vu{e}_r \times \vb{p}) = - \vb{p} +  (\vb{p}\cdot\vu{e}_r)\vu{e}_r,
\end{align*}
%
y al calcular el rotacional de $\vu{e}_r\times\vb{p}$, ésta expresión se reduce a\footnote{Si $\vb{a}$ es un vector constante y $\vb{f}(\vb{r})=f(r)\vu{e}_r$, entonces \cite{jackson1999electrodynamics} $$(\vb{a}\cdot\nabla)\vb{f} = [f(r)/r][\vb{a}-\vu{e}_r(\vb{a}\cdot\vu{e}_r)]+\vu{e}_r(\vb{a}\cdot\vu{e}_r)\pdv*{f}{r}.$$}
%
\begin{align*}
\nabla\times (\vu{e}_r\times\vb{p}) &= \vu{e}_r(\nabla\cdot\vb{p}) - \vb{p}(\nabla\cdot\vu{e}_r) + (\vb{p}\cdot\nabla)\vu{e}_r-(\vu{e}_r\cdot\nabla)\vb{p}\\
	&=  - \vb{p}(\nabla\cdot\vu{e}_r) + (\vb{p}\cdot\nabla)\vu{e}_r\\
	&= -\vb{p}\frac{2}{r} + \frac{1}{r}\qty[\vb{p}-\vu{e}_r(\vb{p}\cdot\vu{e}_r)]\\
	&= -\frac{1}{r}[\vb{p}+\vu{e}_r(\vb{p}\cdot\vu{e}_r)].
\end{align*}
%
Sustituyendo estas expresiones y desarrollando la derivada respecto a $r$, el campo eléctrico se reescribe como
%
\begin{align*}
\vb{E} &= \frac{ik e^{ikr}}{4\pi\varepsilon_0}\qty[
	\qty(\frac{1}{r^2}-\frac{1}{ikr^3})[-\vb{p}-\vu{e}_r(\vb{p}\cdot\vu{e}_r)] + 
	\qty(-\frac{2}{r^2}+\frac{ik}{r}+\frac{2}{ikr^3})[-\vb{p}+\vu{e}_r(\vb{p}\cdot\vu{e}_r)]
]\\
	&=  \frac{e^{ikr}}{4\pi\varepsilon_0}\qty[
	\qty(\frac{ik}{r^2}-\frac{1}{r^3})[-\vb{p}-\vu{e}_r(\vb{p}\cdot\vu{e}_r)] + 
	\qty(-\frac{2ik}{r^2}-\frac{k^2}{r}+\frac{2}{r^3})[-\vb{p}+\vu{e}_r(\vb{p}\cdot\vu{e}_r)],
]
\end{align*}
%
y, finalmente, agrupando términos se obtiene la expresión
%
\begin{align*}
\vb{E} = \frac{e^{ikr}}{4\pi\varepsilon_0}\qty[
	\frac{k^2}{r}(\vu{e}_r\times\vb{p})\times\vu{e}_r + 
	[3\vu{e}_r(\vu{e}_r\cdot\vb{p})-\vb{p}](\frac{1}{r^3}-\frac{ik}{r^2})
].
\end{align*}.
%

\item Calcule el vector de Poynting asociado a la extinción [Ec. \eqref{eq:Sext}] para una partícula esparcidora arbitraria iluminada por una onda plana monocromática polarizada en la dirección $x$ y que se propaga en la dirección $z$.
\textit{Hint:} Emplee las Ecs. \eqref{eq:EscaX} y \eqref{eq:HscaX}.
\label{ex:Wext}
\item[\color{blue} Solución:]
Empleando la Ec. \eqref{eq:HscaX} y que $\vb{E}^i = E_0\vu{e}_x$, entonces
%
\begin{align*}
\vb{E}^i\times\vb{H}^{s*} &= E_0\vu{e}_x\times\qty(\frac{k}{\omega\mu}\vu{e}_r\times\vb{E}^s)^*\\
	&= E_0 \vu{e}_x\times\qty[\qty(\frac{k}{\omega\mu}\vu{e}_r)\times\qty(\frac{e^{ik(r-z)}}{-ikr}E_0\vb{X})]^*\\
	&= \frac{k}{\mu\omega}\frac{E_0^2}{ikr}e^{-ik(r-z)}\big[\vu{e}_x\times(\vu{e}_r\times\vb{X}^*)\big]\\
&= \frac{k}{\mu\omega}\frac{E_0^2}{ikr}e^{-ik(r-z)}
	\big[
	\vu{e}_r(\vu{e}_x\cdot\vb{X}^*)-\vb{X}^*(\vu{e}_x\cdot\vu{e}_r)\big],
\end{align*}
%
y análogamente con la Ec. \eqref{eq:EscaX}
%
\begin{align*}
\vb{E}^s\times\vb{H}^{i*} = \frac{k}{\mu\omega} \frac{E_0^2}{-ikr}\big[
	\vu{e}_z(\vb{X}\cdot\vu{e}_x)-\vu{e}_x(\vb{X}\cdot\vu{e}_z)\big],
\end{align*}
%
por lo que
%
\begin{align*}
\vb{S}^{ext} = \frac{kE_0^2}{2\mu\omega}\Re\bigg\{&
\frac{e^{-ikr}}{ikr}e^{ikz}
	\big[
	\vu{e}_r(\vu{e}_x\cdot\vb{X}^*)-\vb{X}^*(\vu{e}_x\cdot\vu{e}_r)		
	\big]\\
+&
\frac{e^{ikr}}{-ikr}e^{-ikz}
	\big[
	\vu{e}_z(\vb{X}\cdot\vu{e}_x)-\vu{e}_x(\vb{X}\cdot\vu{e}_z)
	\big]
\bigg\}.
\end{align*}
%

Dado que $\vb{X}\cdot\vu{e}_r = 0$, $\vu{e}_r\cdot\vu{e}_z = \cos\theta$ y $\vb{e}_r\cdot\vb{e}_x = \cos\varphi\sin\theta$, la energía que cruza una superfice  esférica $A$ de radio $r$ $W_{ext}$ es
%
\begin{align*}
W_{ext} = -\int_{A} \vb{S}^{ext}\cdot\vu{e}_r\dd{a}
 = \frac{kE_0^2}{2\mu\omega}\Re\bigg\{&
\frac{e^{-ikr}}{-ikr}
	\int_A e^{ikz} (\vu{e}_x\cdot\vb{X}^*)\dd{a}\\
+&\frac{e^{ikr}}{ikr}
	\int_A e^{-ikz}(\vb{X}\cdot\vu{e}_x)\cos\theta\dd{a}\\
-&\frac{e^{ikr}}{ikr}
	\int_A e^{-ikz}(\vb{X}\cdot\vu{e}_z)\sin\theta\cos\varphi\dd{a}
	\bigg\}.
\end{align*}
%
\end{enumerate}
