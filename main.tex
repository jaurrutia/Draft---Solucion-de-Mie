\documentclass[12pts,a4paper]{book}

\usepackage{lipsum}

<<<<<<< HEAD
=======
\usepackage[style = trad-abbrv, backend = bibtex, sorting = none, backref=true]{biblatex} %style = trad-abbrv
%		\addbibresource{06-Bibliografia/references.bib}
\addbibresource{references.bib}
>>>>>>> 65711adc10d96288917d1aaba2996bf656bbde5a
\input{0-PackagesAndCommands/packages.tex}

\title{Solución de Mie}
\author{Urrutia Anguiano, Jonathan Alexis\\ Rojas, Isabel}

% ---------------------------For graphics / images
\usepackage{graphicx}
\graphicspath{{graphics/}}

\makeindex

\begin{document}
	\frontmatter
	
%	\blankpage
%	r.3 full title page
	\maketitle
%	r.5 contents
	\tableofcontents
%	\listoffigures
%	\listoftables

	\cleardoublepage
%\chapter*{Introduction}
%
%Aquí iría el texto que todos vamos a escribir al final, en donde hablamos del propósito del libro y esas cosas. Esto lod dejaremos al final.


% Start the main matter (normal chapters)
\mainmatter


\chapter{Repaso de electromagnetismo}
\label{ch:repasoEM} %-------------------Labels para los capítulos -> ch:LABEL

<<<<<<< HEAD
	El contenido propuesto es

	\begin{itemize}
	 \item Ecuaciones de Maxwell
	 \item Considerar la Ec. de Helmholts
	 \item Hablar sobre los campos EM armónicos
	 \item Vector de Poynting
	 \item Condiciones a la frontera de forma general
	 \item Condiciones a la frontera considerando una esfera
	\end{itemize}
=======
\input{1-Capitulo-Repaso/repaso_electro.tex}

%Esta parte corresponde a \textbf{Isabel}. El contenido propuesto es

%\begin{itemize}
% \item Ecuaciones de Maxwell
% \item Considerar la Ec. de Helmholts
% \item Hablar sobre los campos EM armónicos
% \item Vector de Poynting
% \item Condiciones a la frontera de forma general
% \item Condiciones a la frontera considerando una esfera
%\end{itemize}
>>>>>>> 65711adc10d96288917d1aaba2996bf656bbde5a

\chapter{Teoría general de esprcimiento}
\label{ch:EsparcimientoGral}

	El contenido propuesto es

	\begin{itemize}
	 \item Matriz de esparcimiento general
	 \item Teorema óptico
	\end{itemize}

\chapter{Los armónicos esféricos vectoriales}
\label{ch:ArmonicosEsferico} 
	%Esta parte corresponde a \textbf{Jonathan}. El contenido propuesto es
	%
	%\begin{itemize}
	% \item Propuesta de M y N (gral)
	% \item Función generadora (gral)
	% \item Geometría esférica para M y N
	% \item Geometría esférica para M y N -- Relaciones de ortogonalidad
	%  \item Descomposición de una onda plana en la base de los AEV
	%\end{itemize}
\input{Capitulo-3/ArmonicosEsfericosVectoriales.tex}





\chapter{Esparcimiento y absorción de una esfera}
\label{ch:AEV} %----
Esta parte corresponde a \textbf{Eduardo}. El contenido propuesto es

\begin{itemize}
 \item Citar a las Condiciones a la frontera de la esfera
 \item Cálculo del campo dentro y fuera de la esfera (Coefficientes de Mie)
 \item Matriz de esparcimiento
 \item Análisis de $a_n$, $b_n$ y las $S_{1,2}$
\end{itemize}

\chapter{Casos particulares}
\label{ch:AEV} %----
Esta parte aún no tiene autor asignado. El contenido propuesto es

\begin{itemize}
 \item Gota de agua
 \item Modelo de Drude (material plasmónico)
 \item Oro y plata (con experimento)
 \item Tugsteno (con experimento)
\end{itemize}

<<<<<<< HEAD
\setlength\bibitemsep{.1\itemsep}
=======
%\bibliography{sample-handout}
%\bibliographystyle{plainnat}

>>>>>>> 65711adc10d96288917d1aaba2996bf656bbde5a
\printbibliography

\printindex

\end{document}
